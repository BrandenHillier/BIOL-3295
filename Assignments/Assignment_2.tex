\documentclass[10pt, oneside]{article}   	% use "amsart" instead of "article" for AMSLaTeX format
\usepackage[margin=2cm]{geometry}                		% See geometry.pdf to learn the layout options. There are lots.
\geometry{letterpaper}                   		% ... or a4paper or a5paper or ... 
%\geometry{landscape}                		% Activate for for rotated page geometry
\usepackage[parfill]{parskip}    		% Activate to begin paragraphs with an empty line rather than an indent
\usepackage{graphicx}				% Use pdf, png, jpg, or eps§ with pdflatex; use eps in DVI mode
							% TeX will automatically convert eps --> pdf in pdflatex		
\usepackage{amssymb}
\usepackage{hyperref}
\usepackage{placeins}

%\date{}							% Activate to display a given date or no date



\begin{document}
\textbf{BIOL 3295 Fall 2019}

\textbf{Assignment 2 due Friday Oct 11, 1pm}

\subsubsection*{Geometric and exponential growth}
Note that in class, I refer to exponential growth in discrete time as `geometric growth', while the Vandermeer and Goldberg textbook refers to this as exponential growth.

\begin{description}
\item[1.1] Exercise 1.1 in \cite{vandermeer}. Calculate the population size for $t = 1, 2, 3, 4$ and 5. The questions states that you should assume $N_0=1$.
\item[1.2] Exercise 1.2 in \cite{vandermeer}, but only for $\lambda = 2$.
\item[1.3] Exercise 1.3 in \cite{vandermeer}. Do this question only for tripling time. Here, `exponentially growing' refers to a discrete time geometric growth equation (equation 3 in Vandermeer and Goldberg, 2013). Note that the population has tripled when $N_{t}/N_0 = 3$. Use the general solution: $N_t = \lambda^t N_0$, and take the natural logarithm of both sides of an equation to isolate $t$ in your formula. Come see myself or a TA if you need help. 
\item[1.4] Exercise 1.4 in \cite{vandermeer}. Note that your plot will consist of 12 points; the four points: $(N_1, N_2)$, $(N_2, N_3)$, $(N_3, N_4)$, and $(N_4, N_5)$ where $(x,y)$ are the x- and y- coordinates of the point; for each of the three $\lambda$ values. Use different symbols to represent the 3 different $\lambda$ values.
\item[1.5] Assume a population is growing exponentially (continuous time). Let $r=1.5$ and $N(0)=1$. Calculate the population size at time, $t=5$. 
\item[1.6] Assume a population is growing exponentially (continuous time). Let $r=1.5$ and $N(0)=1$. What is the rate of change in population size, $\frac{dN(t)}{dt}$?
\end{description}

\subsubsection*{Logistic growth}
\begin{description}
\item[2.1] The continuous time logistic growth equation is equation 17 in \cite{vandermeer}. For this equation, for what values of $N$, is $\frac{dN(t)}{dt}=0$?
\item[2.2] Sketch a graph of the solution to a continuous time logistic growth model (i.e. such that $\frac{dN}{dt} = rN\left(1-\frac{N}{K}\right)$) with population size, $N$, on the y-axis and time, $t$ on the x-axis. Plot the following scenarios:
\begin{description}
\item[(a)] $r > 0$ and $0<N(0)<K$, where $N(0)$ denotes the population size at time, $t=0$,
\item[(b)] $r > 0$ and $N(0)>K$, and
\item[(c)] $r < 0$ and  $0<N(0)<K$. 
\end{description}
Please make sure your answer clearly indicates which lines correspond to (a),(b), and (c).
\item[2.3] Define the per capita growth rate as $\frac{dN}{dt}\frac{1}{N}$. Sketch a graph of the per capita growth rate for a continuous time logistic model (y-axis) versus population size, $N$ (x-axis). Assume $r>0$ and make sure your graph clearly indicates:
\begin{description}
\item[$\bullet$] The value of the per capita growth rate when $N = 0$ (i.e., the y-intercept).
\item[$\bullet$] The value of $N$ when the per capita growth rate is 0 (i.e., the x-intercept)
\item[$\bullet$] The slope of the line.
\end{description}  
\item[2.4] Name a significant limitation of May's discrete time logistic map potentially limiting it's applicability to biological populations.
\end{description}

\begin{thebibliography}{10}
\bibitem[Vandermeer and Goldberg, 2013]{vandermeer} Vandermeer, J. H. and D. E. Goldberg, 2013. Population ecology: first principles. Princeton University Press. Available as an ebook from the MUN library. \url{https://ebookcentral.proquest.com/lib/mun/detail.action?docID=1205619}
\end{thebibliography}



\end{document}  