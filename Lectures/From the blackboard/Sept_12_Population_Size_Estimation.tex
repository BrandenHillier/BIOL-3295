\documentclass[11pt, oneside]{article}   	% use "amsart" instead of "article" for AMSLaTeX format
\usepackage{geometry}                		% See geometry.pdf to learn the layout options. There are lots.
\geometry{letterpaper}                   		% ... or a4paper or a5paper or ... 
%\geometry{landscape}                		% Activate for for rotated page geometry
\usepackage[parfill]{parskip}    		% Activate to begin paragraphs with an empty line rather than an indent
\usepackage{graphicx}				% Use pdf, png, jpg, or eps§ with pdflatex; use eps in DVI mode
							% TeX will automatically convert eps --> pdf in pdflatex		
\usepackage{amssymb}
\usepackage{hyperref}
\usepackage{placeins}

%\date{}							% Activate to display a given date or no date

\begin{document}

\begin{table}[htp]
\subsection*{Estimating population abundance for simple random sampling}
\begin{center}
\begin{tabular}{|ll|}
\hline
& \\
$\hat{X}=K\bar{X}$ & the estimated population size. This is formula 8.6 from \cite{Ska} \\
& \\
$K$ & the number of sampling units in the landscape. \\
& \\
$\bar{X} = \frac{1}{k} \sum_{i=1}^k x_i$ & the mean count per sampling unit. \\
& \\
$k$ & the number of sampling units in the landscape. \\
& \\
$x_i$ & the count in the $i^{th}$ sampling unit. \\
& \\
$\hat{X} \pm t_{k-1, 1-\frac{\alpha}{2}} SE(\hat{X})$ & is the $100(1-\alpha)$ percent confidence interval. This is \\
 &  equation 8.13 from \cite{Ska}.\\
 & \\
 $t_{k-1, 1-\frac{\alpha}{2}}$ & the t-statistic with $k-1$ degrees of freedom, and where $\alpha=0.05$\\
 & will evaluate the 95$\%$ confidence interval.\\
 & \\
 $SE(\hat{X}) = \sqrt{\frac{K^2}{k}\left(1-\frac{k}{K}\right)s^2}$ & the standard error in the population size estimate. This is equation \\
 &  8.8 in \cite{Ska}.\\
 & \\
 $s^2 = \frac{\sum_{i=1}^k (x_i -\bar{X})^2}{k-1}$ & between-sampling unit variability. \\
 & \\
 $\hat{\sigma} = SE(\hat{X})\sqrt{k}$ & estimated standard error.\\ \hline
\end{tabular}
\end{center}
\label{default}
\end{table}

\subsection*{Example}
Let,
\[x_1 = 40 \qquad x_2 = 30 \qquad x_3 = 20 \qquad k=3 \qquad \mbox{and}\qquad K=10.\]

The estimated population abundance is,

\[\hat{X} = K \bar{X} = 10 \times \frac{1}{3}(40 + 30 + 20) = 10 \times 30 = 300. \]

We can use \texttt{R} or a t-table to calculate $t_{2, 0.975}$ as,

\[\texttt{>qt(0.975,2) = 4.30}.\]

We estimate the between-sampling unit variability as,

\[s^2 = \frac{(40-30)^2 + (30-30)^2 + (20 - 30)^2}{3-1} = \frac{(10)^2 + (0)^2 + (-10)^2}{2}=100,\]

and the standard error in the population estimate as,

\[SE(\hat{X}) = \sqrt{\frac{10^2}{3}\left(1-\frac{3}{10}\right)100} = 10\sqrt{\frac{100}{3}\times 0.7}=48.3.\]

As such, the lower bound on $95\%$ confidence interval is,

\[300 - 4.3 \times 48.3 = 92,\]

and the upper bound on $95\%$ confidence interval is,

\[300 + 4.3 \times 48.3 = 508.\]

A Q-Q plot can be used evaluate where the $x_i$'s are normally distributed as assumed by the calculations.

\subsection*{Sample size calculation}
\begin{table}[htp!]
\begin{center}
\begin{tabular}{|ll|}
\hline
& \\
$k^*= \left( \frac{Z_{1-\frac{\alpha}{2}} \ \hat{\sigma}}{\delta} \right)^2$ & the minimum simple size needed to achieve a $100(1-\alpha)$ percent \\ 
&  confidence interval with a half-width equal to $\delta$. This is equation 2 from \cite{Greg}.  \\
& \\
$Z_{1-\frac{\alpha}{2}}$ & the value for which the cumulative density of the standard normal  \\
& distribution is $1-\frac{\alpha}{2}$.\\
$\hat{\sigma} = SE(\hat{X})\sqrt{k}$ & the estimated standard deviation of the population estimate. \\ \hline
\end{tabular}
\end{center}
\label{default}
\end{table}
\FloatBarrier

We evaluate $Z_{1-\frac{\alpha}{2}}$ in \texttt{R} as \texttt{>qnorm(1-$\frac{\alpha}{2}$)}. Therefore, if $\alpha = 0.05$ we calculate $k^*$ for a 95$\%$ confidence interval as,

\[\texttt{>qnorm(0.975) = 1.96}. \]

The estimated standard deviation is,

\[\hat{\sigma} = 48.3 \times \sqrt{3} = 83.7,\]

where $SE(\hat{X}) = 48.3$ and $k = 3$ are the values from the previous section.

If we want $\delta = 50$ then,

\[k^* = \left(\frac{1.96 \times 83.7}{50} \right)^2 = 10.7.\]

Therefore, since $K=10$ and for a landscape that is as variable as this one ($\hat{\sigma} = 83.7$),  we could only get a 95$\%$ confidence interval with a halfwidth of $\delta = 50$ if we sampled all the sampling units in the landscape. 

\begin{thebibliography}{1}
\bibitem{Greg} Gregoire, T.G., and D. L. R. Affleck. 2018. Estimating Desired Sample Size for Simple Random Sampling of a Skewed Population, The American Statistician, 72:2, 184-190, \url{DOI: 10.1080/00031305.2017.1290548}. 
\bibitem{Ska} Skalski, J.R., Ryding, K.E., Millspaugh, J.J., Millspaugh, J., 2005. Wildlife Demography: Analysis of
Sex, Age, and Count Data. Elsevier Science and Technology, Burlington, UNITED STATES.
\url{https://ebookcentral-proquest-com.qe2a-proxy.mun.ca/lib/mun/detail.action?docID=269552}
\end{thebibliography}



\end{document}  